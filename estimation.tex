%
%\begin{algorithm}[]
%	\DontPrintSemicolon
%
%	\caption{BicliqueEstimation2($\mathbb{S},p,q,\epsilon,\delta$)}
%	\label{BicliqueEstimation}
%	\KwIn{ Shadow $\mathbb{S} $ integers $p,q $, error parameters$\epsilon,\delta$ and a rough esitmation for the $(p,q)$-biclique density $  \tilde{\mu}$ }
%	\KwOut{$\tcnt_{(p,q)}(G)$ }
%
%	$s \gets 0; s_{est} \gets 0; t \gets 0$ \\
%	$t_b \gets \frac{\gamma}{\tilde{\mu}}$ \\
%	constuct alias structure for sampling sub spaces \\
%	\While{ $s < \gamma $   }{
%
%%		\For{\textbf{each} subsapce ($R_U,R_V,S_U,S_V) \in \mathbb{S}$} {
%%			initialize $c_{(R_U,R_V,S_U,S_V)} \gets 0$
%%		}
%	sample a subspace (($R_U,R_V,S_U,S_V,\ddot{\mu}))  \in \mathbb{S}$ with probability $\frac{\mathcal{P}_{(p-|R_U|,q-|R_V|)}(S_U,S_V)}
%	{|\samplespace|}$
%%		\For{$i \gets 1$ to $t_b$}{
%%
%%
%%			$c_{(R_U,R_V,S_U,S_V)} \gets   c_{(R_U,R_V,S_U,S_V)} + 1 $
%%		}
%
%		\For{\textbf{each} subsapce ($R_U,R_V,S_U,S_V) \in \mathbb{S}$ with  $c_{(R_U,R_V,S_U,S_V)} > 0$} {
%			\red{how to construct the alias structure here}
%
%			\For{$i \gets 1$ to $c_{(R_U,R_V,S_U,S_V)} > 0$}{
%				sample an element P u.a.r. from  $|\mathcal{P}_{(p-|R_U|,q-|R_V|)}(S_U,S_V)$\\
%
%				\If{ P formas a biclique in G} {
%					$s \gets s + C_{(p-|R_U|,q-|R_V|)} /$ \red{todo here}
%					$s_{est} \gets s_{est} + C_{(p-|R_U|,q-|R_V|)} $
%				}
%				$t \gets t + 1$
%			}
%		}
%
%	}
%	\red{correct this}\\
%	$\tcnt_{(p,q)}(G) \gets  |\samplespace|  \cdot \frac{s}{t} $\\
%	\Return{$\tcnt_{(p,q)}(G) $}
%\end{algorithm}


\subsection{Estimating Bicliques}



\begin{algorithm}[]
	\DontPrintSemicolon

	\caption{BicliqueEstimation2($\mathbb{S},p,q,\epsilon,\delta$)}
	\label{bcefinal}
	\KwIn{ Shadow $\mathbb{S} $ integers $p,q $, error parameters $\epsilon,\delta$ }
	\KwOut{$\tcnt_{(p,q)}(G)$ }

$s \gets 0;\ t \gets 0$\;

$\gamma = 1 + \frac{4(1+\epsilon)(e -2)\ln(2/\delta)}{\epsilon^2}$\;

\While{$s < \gamma$}{
	Sample a subspace $(R_U, R_V, S_U, S_V, \ddot{\mu}) \in \mathbb{S}$ with probability $\dfrac{|\mathcal{P}_{(p - |R_U|, q - |R_V|)}(S_U, S_V)|}{|\sampelspace|}$\;

	Sample an element $P$ uniformly at random from $\mathcal{P}_{(p - |R_U|, q - |R_V|)}(S_U, S_V)$\;

	\If{$P$ forms a biclique in $G$}{
		$s \gets s + 1$\;
	}
	$t \gets t + 1$\;
}
$\estcnt \gets |\sampelspace| \cdot \dfrac{s}{t}$\;
\Return{$\estcnt$}

\end{algorithm}



The final algorithm for estimating the number of $(p,q)$-bicliques is shown in Algorithm~\ref{bcefinal}.The algorithm proceeds by repeatedly sampling elements from the overall sample space until the number of successful samples $s$ reaches the threshold $\gamma$ (Lines 3 8). In each iteration, a subspace $(R_U, R_V, S_U, S_V, \ddot{\mu})$ is selected from the shadow $\mathbb{S}$ with probability proportional to the size of its corresponding sampling space $\mathcal{P}_{(p - |R_U|, q - |R_V|)}(S_U, S_V)$ relative to the total sample space size $|\sampelspace|$ (Line 4). This weighted sampling ensures that every potential $(p,q)$-biclique in the total sample space has an equal probability of being selected, maintaining uniformity across the entire sample space.

Once a subspace is selected, an element $P$ is sampled uniformly at random from $\mathcal{P}_{(p - |R_U|, q - |R_V|)}(S_U, S_V)$ (Line 5). This element represents a $p-|R_U|$-zstar.We then check whether the sampled element $P$ forms a biclique in $G$ (Line 6). If it does, we increment the successful sample counter $s$ (Line 7). Regardless of whether $P$ forms a biclique, we increment the total sample counter $t$ (Line 8). This process continues until the number of successful samples $s$ reaches the threshold $\gamma$ (Line 3), ensuring that sufficient evidence has been gathered to achieve the desired estimation accuracy.After obtaining the required number of successful samples, we compute the estimate $\estcnt$ of the total number of $(p,q)$-bicliques (Line 9). This estimate is calculated by scaling the ratio of successful samples to total samples by the size of the overall sample space $|\sampelspace|$.Finally, the algorithm returns this estimate $\estcnt$ (Line 10), which, due to the properties of the sampling method and the stopping condition, satisfies the accuracy and confidence guarantees specified by the error parameters $\epsilon$ and $\delta$.

