
% VLDB template version of 2020-08-03 enhances the ACM template, version 1.7.0:
% https://www.acm.org/publications/proceedings-template
% The ACM Latex guide provides further information about the ACM template

\documentclass[sigconf, nonacm]{acmart}

\usepackage{amsmath}
\usepackage{import}
\usepackage[utf8]{inputenc}
\usepackage{mathrsfs,amsfonts}
\usepackage{tikz-network}
\usepackage{esvect}
\usepackage[]{geometry}
\usepackage{float}
\usepackage{color,soul}
\usepackage{subfigure}
\usepackage{subfiles}
\usepackage{import}
%%%%%%%%%%%%%% Modified Algorithm Package %%%%%%%%%%%%%%%%%
\let\algorithm\relax
\let\endalgorithm\relax

\usepackage[
%algochapter,
resetcount,
algoruled,
%longend,
linesnumbered,
%lined
vlined
]{algorithm2e}

\SetKw{OR}{or}
\SetKw{AND}{and}
\SetKw{NOT}{not}

\SetKw{TRUE}{true}
\SetKw{FALSE}{false}
\SetKw{NULL}{nil}

\SetKw{KwDownto}{downto}

% the default [longend] texts really suck, so here are my definitions
\SetKwIF{gIf}{gElsIf}{gElse}{if}{then}{else if}{else}{end if}%
\SetKwIF{If}{ElseIf}{Else}{if}{then}{else if}{else}{end if}%
\SetKwFor{For}{for}{do}{end for}%
\SetKwFor{ForPar}{for}{do in parallel}{end for}%
\SetKwFor{ForEach}{for each}{do}{end for}%
\SetKwFor{ForAll}{forall}{do}{end for}%
\SetKwFor{While}{while}{do}{end while}%
\SetKw{Break}{break\;}%

\newcommand{\State}[1]{#1\;}
\newcommand{\StateCmt}[2]{% statement (#1) with comment (#2)
	#1; \tcc*[f]{#2}\;
}
%% this command is not well implemented (buggy or ugly)
\newcommand{\CmtState}[2]{% comment (#2) and then statement (#1)
	\tcc*[f]{#2} #1\;
}

\newcommand{\Comment}[1]{%
	\tcc{\footnotesize #1}
}

%% global (default) setting
\SetKwInOut{Input}{Input}
\SetKwInOut{Output}{Output}
\SetKwInput{KwDescription}{Description}

%% change the default style of \Return: (1) make text appearing in normal font, and (2) add ;
\renewcommand{\Return}[1]{\State{\textbf{return} #1}}

\let\oldnl\nl% Store \nl in \oldnl
\newcommand{\nonl}{\renewcommand{\nl}{\let\nl\oldnl}}% Remove line number for one line

%%%%%%%%%%%%%%%%%%% End of Modified Algorithm Package %%%%%%%%%%%%%%%%%

%\newcommand{\nthesection}{\arabic{section}}

%\newcounter{example}[section]
%\renewcommand{\theexample}{\nthesection.\arabic{example}}
%\newenvironment{example}{
	%     \refstepcounter{example}
	%     {\goodbreak\vspace{1ex} \noindent\bf  Example  \theexample:}}{
	%     \eot\vspace{1ex}\goodbreak} %\hspace*{\fill}\vspace*{1ex}}

	%\newcounter{definition}[section]
	%\renewcommand{\thedefinition}{\nthesection.\arabic{definition}}
	%\newenvironment{definition}{
%     \refstepcounter{definition}
%     {\goodbreak\vspace{1ex} \noindent\bf Definition  \thedefinition:}}{
%     \hspace*{\fill}\vspace*{1ex}\goodbreak}

%\newcounter{theorem}[section]
%\renewcommand{\thetheorem}{\nthesection.\arabic{theorem}}
%\newenvironment{theorem}{\begin{em}
	%        \refstepcounter{theorem}
	%        {\goodbreak\vspace{1ex} \noindent\bf  Theorem  \thetheorem:}}{
	%        \end{em} \hspace*{\fill}\vspace*{1ex}\goodbreak}

	%\newcounter{lemma}[section]
	%\renewcommand{\thelemma}{\nthesection.\arabic{lemma}}
	%\newenvironment{lemma}{\begin{em}
	%        \refstepcounter{lemma}
	%        {\goodbreak\vspace{1ex}\noindent\bf Lemma \thelemma:}}{
	%		\end{em} \hspace*{\fill}\vspace*{1ex}\goodbreak}

	%\newcounter{property}[section]
	%\renewcommand{\theproperty}{\nthesection.\arabic{property}}
	%\newenvironment{property}{\begin{em}
	%        \refstepcounter{property}
	%        {\goodbreak\vspace{1ex}\noindent\bf Property \theproperty:}}{
	%		\end{em} \hspace*{\fill}\vspace*{1ex}\goodbreak}

	%\newcounter{corollary}[section]
	%\renewcommand{\thecorollary}{\nthesection.\arabic{corollary}}
	%\newenvironment{corollary}{\begin{em}
	%        \refstepcounter{corollary}
	%        {\goodbreak\vspace{1ex}\noindent\bf Corollary \thecorollary:}}{
	%		\end{em} \hspace*{\fill}\vspace*{1ex}\goodbreak}


	\newcommand{\proofsketch}{\noindent{\bf Proof Sketch: }}
	\newcommand{\myproof}{\noindent\textbf{\em Proof: }}


	% End of proof
	\newcommand{\eot}{\hspace*{\fill}\mbox{$\Box$}}
	% \newcommand{\eop}{\hspace*{\fill}\mbox{$\blacksquare$}}
	\newcommand{\eop}{\hspace*{\fill}\mbox{$\Box$}\vspace*{1ex}\goodbreak}

	% Small Title
	\newcommand{\stitle}[1]{\vspace{1ex} \noindent{\bf #1}}

	\newcommand{\kk}[1]{{\ensuremath {\mathtt{#1}}}\xspace}
	\newcommand{\kw}[1]{{\ensuremath {\mathsf{#1}}}\xspace}

	\newcommand{\la}{\leftarrow}
	\newcommand{\ra}{\rightarrow}

	\newcommand{\cfig}{Figure~}
	\newcommand{\ctab}{Table~}
	\newcommand{\csec}{Section~}
	\newcommand{\cdef}{Definition~}
	\newcommand{\calg}{Algorithm~}
	\newcommand{\cthm}{Theorem~}
	\newcommand{\clem}{Lemma~}
	\newcommand{\cexm}{Example~}
	\newcommand{\cpro}{Property~}
	\newcommand{\cequ}[1]{Equation~(#1)}

	\newcommand{\ie}{{i.e.}}
	\newcommand{\eg}{{e.g.}}

	\newcommand{\cal}{\mathcal}
	\newcommand{\bigo}{ {\cal O}}

	\newcommand{\norm}[1]{\lVert#1\rVert}
	\newcommand{\bicliques}{{$(p,q)$-bicliques }}
		\newcommand{\biclique}{{$(p,q)$-biclique }}
	\newcommand{\estcnt}{{\widehat{cnt}_{p,q}(G) }}
		\newcommand{\cnt}{{cnt_{p,q}(G) }}
	\newcommand{\red}[1]{\textcolor{red}{#1}}
	\newcommand{\bicliqedensity}{$\mathcal{\hat{B}} $ }
	\newcommand{\bicliqedensityG}{$\mathcal{\hat{B}}(G)$ }
\newcommand{\sampelspace}{\mathcal{S}_{p,q}(G)}
\newcommand{\shadow}{$\mathbb{S}_{p,q}(G)$ }
\newcommand{\zstars}{$h$-zstars }
\newcommand{\zstar}{$h$-zstar }
\newcommand{\z}{$Z^* $ }
\newcommand{\zstarpq}{Z_{(p,q)} }
\newcommand{\bicliquecount}{$cnt_{(p,q)}$}
\newcommand{\tcnt}{\widetilde{cnt}}
\newcommand{\setbiclique}{\mathcal{B}_{p,q}}
\newcommand{\setzstar}{\Delta (G)}
\newcommand{\bicliquedensityshadow}{\mu_{\mathcal{S}{p,q}}}

\newcommand{\subspace}{(R_U, R_V, S_U, S_V)}
















\usepackage{graphicx}
\usepackage{multirow}
\usepackage{pdfpages}
\usepackage{appendix}
\usepackage{xcolor}
\usepackage{booktabs}
\usepackage{tikz}
%% The following content must be adapted for the final version
% paper-specific
\newcommand\vldbdoi{XX.XX/XXX.XX}
\newcommand\vldbpages{XXX-XXX}
% issue-specific
\newcommand\vldbvolume{14}
\newcommand\vldbissue{1}
\newcommand\vldbyear{2020}
% should be fine as it is
\newcommand\vldbauthors{\authors}
\newcommand\vldbtitle{\shorttitle}
% leave empty if no availability url should be set
\newcommand\vldbavailabilityurl{URL_TO_YOUR_ARTIFACTS}
% whether page numbers should be shown or not, use 'plain' for review versions, 'empty' for camera ready
\newcommand\vldbpagestyle{plain}




\begin{document}
\title{A Relational Model of Data for Large Shared Data Banks}

%%
%% The "author" command and its associated commands are used to define the authors and their affiliations.
\author{Ben Trovato}
\affiliation{%
  \institution{Institute for Clarity in Documentation}
  \streetaddress{P.O. Box 1212}
  \city{Dublin}
  \state{Ireland}
  \postcode{43017-6221}
}
\email{trovato@corporation.com}


%%
%% The abstract is a short summary of the work to be presented in the
%% article.
\begin{abstract}
write abstract
\end{abstract}

\maketitle

%%% do not modify the following VLDB block %%
%%% VLDB block start %%%
\pagestyle{\vldbpagestyle}
\begingroup\small\noindent\raggedright\textbf{PVLDB Reference Format:}\\
\vldbauthors. \vldbtitle. PVLDB, \vldbvolume(\vldbissue): \vldbpages, \vldbyear.\\
\href{https://doi.org/\vldbdoi}{doi:\vldbdoi}
\endgroup
\begingroup
\renewcommand\thefootnote{}\footnote{\noindent
This work is licensed under the Creative Commons BY-NC-ND 4.0 International License. Visit \url{https://creativecommons.org/licenses/by-nc-nd/4.0/} to view a copy of this license. For any use beyond those covered by this license, obtain permission by emailing \href{mailto:info@vldb.org}{info@vldb.org}. Copyright is held by the owner/author(s). Publication rights licensed to the VLDB Endowment. \\
\raggedright Proceedings of the VLDB Endowment, Vol. \vldbvolume, No. \vldbissue\ %
ISSN 2150-8097. \\
\href{https://doi.org/\vldbdoi}{doi:\vldbdoi} \\
}\addtocounter{footnote}{-1}\endgroup
%%% VLDB block end %%%

%%% do not modify the following VLDB block %%
%%% VLDB block start %%%
\ifdefempty{\vldbavailabilityurl}{}{
\vspace{.3cm}
\begingroup\small\noindent\raggedright\textbf{PVLDB Artifact Availability:}\\
The source code, data, and/or other artifacts have been made available at \url{\vldbavailabilityurl}.
\endgroup
}
%%% VLDB block end %%%

\textbf{Contributions}. Our key contributions are as follows:

\begin{itemize}
	\item We present the first approximation algorithm with formal accuracy guarantees for the $(p,q)$-biclique counting problem.
	\item We propose an innovative edge-oriented technique for refining the sampling space, strategically balancing the computational load between the two stages of the algorithm to optimize overall runtime.
	\item We introduce a novel sampling method that not only ensures accuracy but also allows for dynamic refinement of the sampling process, adapting to the graph’s structure for improved efficiency.
\end{itemize}

These contributions significantly advance the field of biclique counting, offering a practical and efficient approach for large-scale bipartite graph analysis.



\section{Introduction}

%Biclique counting in bipartite graphs, which involves identifying and enumerating complete bipartite subgraphs, is a crucial task in network analysis. Given a bipartite graph \( G = (U, V, E) \) with disjoint vertex sets \( U \) and \( V \), and edge set \( E \subseteq U \times V \), a biclique is defined as a complete bipartite subgraph \( K_{a,b} \), where every vertex in \( U' \subseteq U \) is connected to every vertex in \( V' \subseteq V \). The problem of biclique counting entails determining the number of such \( K_{a,b} \) subgraphs for various values of \( a \) and \( b \).
%
%Bicliques are particularly useful in a variety of applications, including community detection in social networks, motif finding in biological networks, and frequent itemset mining in transaction databases. However, the problem of counting bicliques becomes computationally infeasible for large graphs due to the combinatorial nature of the problem. The number of potential bicliques grows exponentially with the size of the vertex sets, leading to significant challenges in both exact and approximate counting methods.

\subsection{Exact Biclique Counting}
%
%Exact biclique counting algorithms typically rely on exhaustive search methods, which enumerate all possible subsets of vertices and verify the completeness of the subgraph. One classical approach involves iterating over all pairs of subsets \( U' \subseteq U \) and \( V' \subseteq V \), checking whether every vertex in \( U' \) is connected to every vertex in \( V' \). The time complexity of this naive approach is \( O(2^{|U|} \times 2^{|V|}) \), which is infeasible for large graphs.
%
%More sophisticated exact algorithms, such as those based on recursive backtracking and dynamic programming, aim to reduce the search space by exploiting the structural properties of the graph. For instance, algorithms using degeneracy ordering or vertex pruning techniques can significantly reduce the number of candidate subgraphs, improving the practical performance. Despite these improvements, the worst-case time complexity of exact bi clique counting algorithms remains exponential in the size of the graph.

\subsection{Approximation Algorithms}

%Given the computational hardness of exact biclique counting, approximation algorithms have been developed to provide scalable solutions for large graphs.
%
%Another approximation technique involves relaxing the completeness condition of bicliques, allowing the identification of dense bipartite subgraphs that approximate bicliques. These methods leverage techniques from graph clustering and matrix factorization, providing a trade-off between accuracy and computational efficiency.
%
%The time complexity of approximation algorithms varies depending on the specific method employed. For instance, sampling-based methods typically run in \( O(k \times |E|) \), where \( k \) is the number of samples, making them more feasible for large-scale graphs compared to exact algorithms.
%
%\
\section{preliminaries}

In this paper, we focus on a large unweighted and undirected bipartite graph $G = (U, V, E)$, where $U$ and $V$ are sets of vertices, and $E$ is the set of undirected edges. For each vertex $u \in U$ (or $V$), its neighbors are denoted by $N(u) = \{v \mid e(u,v) \in E\}$. The common neighbors of a vertex set $S$ are denoted as $N(S) = \bigcap_{u \in S} N(u)$. Let $d(u)$ be the degree of a vertex $u$, i.e., $d(u) = |N(u)|$.

For simplicity, we assume without loss of generality that the vertices on each side of the bipartite graph $G'$ are arranged in a particular order. Specifically, we order the vertices as $u_1 \prec u_2 \prec \dots \prec u_{n_1}$ for $U$ and $v_1 \prec v_2 \prec \dots \prec v_{n_2}$ for $V$. Additionally, we assume that $p \leq q$, with $p$ representing the number of vertices from the $U$ side and $q$ from the $V$ side. The same methods can be applied when $p > q$.

Please note that, for convenience, the examples provided in this paper consider \red{the natural ordering} of the vertices.

\begin{definition}[Biclique]
A biclique $g=(U',V',E')$ in the bipartite graph $G = (U, V, E)$ is a pair of vertex subsets $U' \subseteq U$ and $V' \subseteq V$ such that every vertex $u \in U'$ is adjacent to every vertex $v \in V'$. \ie,
$E' = \{ (u, v) \mid u \in U', v \in V' \} \subseteq E$.

\end{definition}

A biclique $B$ is called a $(p,q)$-biclique if $|U'|=p$ and $|V'|=q$.

\stitle{Problem Statement} given a large bipartite graph $G$, and two integers $p,q$, and error parameter $\epsilon \in (0,1)$ and $\delta \in (0,1)$,we study the problem of approximately counting the number of \bicliques in $G$. \ie  our main goal is to design a randomized algorithm that outputs an estimated value $\estcnt$ satisfying,

\[
P \left(\Big|\widehat{\text{cnt}}_{(p,q)}(G) - \text{cnt}_{(p,q)}(G)\Big| > \epsilon \cdot \text{cnt}_{(p,q)}(G)\right) \leq \delta
\]



\begin{center}
\begin{figure}

\begin{tikzpicture}[scale=1, 
	every node/.style={circle, draw, minimum size=0.5cm, inner sep=0pt, line width=0.8pt}
]
	
	% Nodes
	\node (u1) at (0,1.5) {$u_1$};
	\node (u2) at (1.3,1.5) {$u_2$};
	\node (u3) at (2.6,1.5) {$u_3$};
	\node (u4) at (3.9,1.5) {$u_4$};

	
	\node (v1) at (0,0) {$v_1$};
	\node (v2) at (1.3,0) {$v_2$};
	\node (v3) at (2.6,0) {$v_3$};
	\node (v4) at (3.9,0) {$v_4$};
	\node (v5) at (5.2,0) {$v_5$};
	
	% Edges
	\draw (u1.south) -- (v1.north);
	\draw (u1.south) -- (v2.north);
	\draw (u1.south) -- (v3.north);
	\draw (u1.south) -- (v4.north);
	\draw (u2.south) -- (v1.north);
	\draw (u2.south) -- (v2.north);
	\draw (u2.south) -- (v3.north);
		\draw (u2.south) -- (v4.north);
	
	\draw (u3.south) -- (v2.north);
	\draw (u3.south) -- (v3.north);
	
	\draw (u3.south) -- (v4.north);
		\draw (u3.south) -- (v5.north);
	

	\draw (u4.south) -- (v4.north);
	\draw (u4.south) -- (v5.north);

\end{tikzpicture}
\caption{An example graph}
\label{fig:ex1} 
\end{figure}
\end{center}
\red{change () to \{\}}\\

Consider the graph shown in Figure \ref{fig:ex1}, there are five (2,3)-bicliques
 $ \mathcal{B}_{(2,3)} = \{\{(u_1, u_2), (v_1, v_2, v_3)\}, \{(u_1, u_2), (v_1, v_2, v_4)\},\{(u_1, u_2), (v_1,\\ v_3, v_4)\},\{(u_1, u_2), (v_2, v_3, v_4)\}\},\{(u_2, u_3), (v_2, v_3, v_4)\}\} $.

%\begin{table}[htb]
%	\small
%	\caption{Frequently Used Notation}
%	\label{tab:notation}
%	\begin{tabular}{c|l} \hline
%		\textbf{Notation} & \textbf{Description} \\ \hline
%		$\mathcal{B}_{(p,q)}$ & Set of all $(p,q)$-bicliques in $G$ \\
%		$\cnt$ & Total number of $(p,q)$-bicliques in $G$, i.e., $\cnt = |\mathcal{B}_{(p,q)}|$ \\
%		$\estcnt$ & Estimated total number of $(p,q)$-bicliques in $G$ \\
%		$\Delta$ & Set of all $(p,q)$-zstars in $G$ \\
%		$\mathcal{\hat{B}}(G)$ & Biclique density of $G$, defined as $\mathcal{\hat{B}}(G) = \frac{\cnt}{|\Delta|}$ \\
%		$Z^*$ & A $(p,q)$-zstar in $G$ \\
%		$\sampelspace$ & Sample space used for estimating $\cnt$ \\
%
%		$\mathbb{S}_{(p,q)}(G)$ & The BC-Shadow sample space for $(p,q)$-bicliques in $G$ \\
%		$\mu_{\sampelspace}$ & Biclique density of the sample space $\sampelspace$, i.e., $\mu_{\sampelspace} = \frac{\cnt}{|\sampelspace|}$ \\
%		$\mathcal{P}_{(p,q)}(S_U, S_V)$ & Set of all pairs $(U', V')$ with $U' \subseteq S_U$, $V' \subseteq S_V$, $|U'|=p$, $|V'|=q$ \\
%		$N(u)$ & Neighbors of vertex $u$, i.e., $N(u) = \{ v \mid (u, v) \in E \}$ \\
%		$N(S)$ & Common neighbors of a vertex set $S$, i.e., $N(S) = \bigcap_{u \in S} N(u)$ \\
%		$d(u)$ & Degree of vertex $u$, i.e., $d(u) = |N(u)|$ \\
%		$\epsilon, \delta$ & Error parameters for approximation, with $\epsilon \in (0,1)$ and $\delta \in (0,1)$ \\
%		$\gamma$ & Parameter used in sampling algorithms, related to $\epsilon$ and $\delta$ \\
%		\hline
%	\end{tabular}
%\end{table}

\begin{table}[htb]
	\small
	\caption{Frequently used notations}
	\label{table:notations}

	\begin{tabular}{c|l} \hline
		Notation & Meaning \\ \hline
	$\setbiclique$ & Set of all $(p,q)$-bicliques in $G$ \\
	$\cnt$ & Total number of $(p,q)$-bicliques in $G$, i.e., $|\mathcal{B}_{(p,q)}|$ \\
	$\estcnt$ & Estimated total number of $(p,q)$-bicliques in $G$ \\
	$\Delta(G)$ & Set of all $(p,q)$-zstars in $G$ \\
	$\sampelspace$ & sampling space of $\setbiclique$ satisfying $\sampelspace \supseteq \setbiclique$ \\
	\shadow & BC-Shadow \\

	\end{tabular}
\end{table}




\section{Our Approach}
\begin{algorithm}[]

	\DontPrintSemicolon
	\caption{BCFramework($G,p,q,\epsilon,\delta$)}

	\KwIn{A bipartite graph $G=(U,V,E)$, integers $p$, $q$ and, accuracy parameters $\epsilon, \delta$}
	\KwOut{an estimation $\estcnt$ for $\cnt$}
		\label{alg:BCFramework}
	Arrange $U$ and $V$ in some order \\
	\scriptsize \tcp{Stage-I: construct a sample space $\sampelspace$, represented by a compact structure $\mathbb{S}$}
	\normalsize $\mathbb{S} \gets  \{(\emptyset,\emptyset,U,V )\}$
	\While{the construction stopping condition is not satisfied}{
		Choose a sample subspace $(R_U,R_V,S_U,S_V)$ and remove it from $\mathbb{S}$\;
		\scriptsize \tcp{Refine the subspace $(R_U,R_V,S_U,S_V)$ by partitioning it}
		\normalsize \For{each $e(u,v)\in S_U \cup S_V$}{
			Add $(R_U \cup u,R_V \cup v, N_{>u}(v) \cap S_U ,N_{>v}(u)\cap S_V)$
		}
	}
	\scriptsize \tcp{Stage-II: sample from \shadow get $\estcnt$}
	\normalsize $|\sampelspace| = \sum_{(R_U, R_V, S_U, S_V) \in \mathbb{S}} |\mathcal{P}_{(p-|R_U|,q-|R_V|)}(S_U,S_V)| $
	$s \leftarrow 0 , t \leftarrow 0$\;

	\While{the sampling stopping condition is not satisfied}{
		Sample a subspace $(R_U,R_V,S_U,S_V)$ from $\mathcal{S}$ with probability $\frac{|\mathcal{P}_{(p-|R_U|,q-|R_V|)}(S_U,S_V)|}
		{|\mathcal{S}_{(p,q)}(G)|}$\;
		Sample an element $P$ u.a.r. from $\mathcal{P}_{(p-|R_U|,q-|R_V|)}(S_U,S_V)$\;
		\If{$P$ forms a \biclique in $G$}{
			$s \leftarrow s + 1$\;
		}
		$t \leftarrow t + 1$\;
	}
	\Return  $\estcnt \gets |\sampelspace| \cdot \frac{s}{t}$

\end{algorithm}





\red{add intro here}\\

Our framework for estimating the $(p,q)$-biclique count is presented in Algorithm \ref{alg:BCFramework} which operates in two stages.

In Stage I (Lines 1–6), we construct a sample space $\mathcal{S}_{(p,q)}(G)$ that meets the following conditions:

\begin{itemize}

\item $\sampelspace$ is a superset of the set of all $(p,q)$-bicliques in $G$, denoted as $\setbiclique$, \ie $\setbiclique \subseteq \sampelspace$.
\item Total count of sample elements in $\sampelspace$ can be computed efficiently.
\item We can efficiently sample elements uniformly at random (u.a.r.) from $\mathcal{S}_{(p,q)}(G)$.

\end{itemize}

 Each element of the sample space $\mathcal{S}_{(p,q)}(G)$ is a pair of vertex sets $(U_p, V_q)$, where $U_p \subseteq U$ and $V_q \subseteq V$, such that $|U_p|=p$ and $|V_q|=q$ with some additional constraints.
In Stage II (Lines 7–12), random samples are drawn from the sample space $\sampelspace$, and the $(p,q)$-biclique count $\cnt$ is estimated as $|\sampelspace|$ multiplied by the fraction of samples that form $(p,q)$-bicliques. Specifically, if $t$ samples are drawn from $\sampelspace$, and for each sample $i$, $X_i = 1$ if the sample forms a $(p,q)$-biclique in $G$, and $X_i = 0$ otherwise, then the estimate of $\cnt$ is given by:

\begin{equation}
	\cnt = |\sampelspace| \cdot \frac{1}{t} \sum_{i=1}^{t} X_i \label{eq}
\end{equation}

The core idea involves defining the $(p,q)$-biclique density of the sample space $\sampelspace$ as:

\begin{equation} \bicliquedensityshadow = \frac{\cnt}{|\sampelspace|} \label{eq
} \end{equation}

This value lies between 0 and 1. Since $\cnt$ is fixed, the $(p,q)$-biclique density depends only on the chosen sample space $\sampelspace$. Different sample spaces will yield different values of $\mu_{\sampelspace}$. Therefore, $\cnt$ can be expressed as:

\begin{equation} \cnt = |\sampelspace| \cdot \mu_{\mathcal{S}_{(p,q)}(G)} \end{equation}

The task then becomes estimating $\mu_{\sampelspace}$, under the assumption that $|\sampelspace|$ can be efficiently determined. Algorithm \ref{alg:BCFramework} estimates $\mu_{\mathcal{S}_{(p,q)}(G)}$ using the empirical mean $\hat{\mu}$ from $t$ random samples, calculated as:

\begin{equation} \hat{\mu} = \frac{1}{t} \sum_{i=1}^{t} X_i \label{eq
} \end{equation}

Thus, the estimate for $\cnt$ becomes:

\begin{equation} \estcnt = |\mathcal{S}_{(p,q)}(G)| \cdot \hat{\mu} \end{equation}

This provides an unbiased estimate for $\cnt$.

\subsection{BC-Shadow}
The core idea of our algorithm, is to sample two vertex sets with from $U$ and $V$ with cardinalities $|p|$ and $|q|$ respectively, and then verify whether these sets form a biclique. A straightforward approach to sample a $(p,q)$-biclique would be to select all possible subsets of $p$ vertices from $U$ and $q$ vertices from $V$, which leads to an unfeasible large sample space for large graphs, i.e., $\binom{|U|}{p} \cdot \binom{|V|}{q}$. Such an approach is computationally impractical due to the excessive memory requirements.To mitigate this, we need to reduce the sample space by eliminating vertex sets that do not form a biclique. In this section, we introduce a novel edge oriented sample space refinement technique, termed \textit{BC-Shadow}, which is inspired by the notion of the shadow that was originally introduced for $k$-clique counting \cite{turanshadow}, was formally defined in \cite{citation}.

\begin{definition}[BC-Shadow]
	Let $G = (U, V, E)$ be a bipartite graph, and let $p$ and $q$ be two integers. The \textbf{BC-Shadow} $\mathbb{S}$ is a quadruple $(R_U, R_V, S_U, S_V)$ representing a sampling space, with the following properties:
	\begin{itemize}
		\item $R_U \subseteq U$ and $R_V \subseteq V$, such that $R_U \cup R_V$ forms a biclique in $G$,
		\item $S_U \subseteq U \setminus R_U$ and $S_V \subseteq V \setminus R_V$, where  $S_U = N(R_V)$, and $S_V = N(R_U)$
		\item For every $(p,q)$-biclique $B_U \cup B_V$ in $G$, there exists a unique subspace $(R_U, R_V, S_U, S_V) \in \mathbb{S}_{p,q}(G)$ such that: $R_U \subseteq B_U, \\  R_V \subseteq B_V ,  B_U \setminus R_U \subseteq S_U, \text{and} B_U \setminus R_V \subseteq S_V.$
	\end{itemize}
\end{definition}

Note that, for counting the number of $(p,q)$-bicliques in $G$, it is enough to store the cardinalities $|R_U|$ and $|R_V|$ within $\mathbb{S}_{p,q}(G)$, rather than explicitly storing the sets $R_U$ and $R_V$. However, if the goal is to sample and report the bicliques, then it is necessary to store the sets $R_U$ and $R_V$ explicitly.

From the definition of the BC-Shadow, the count of $(p,q)$-bicliques in $G$ can be computed as:
\[
\text{cnt}_{p,q}(G) = \sum_{(R_U, R_V, S_U, S_V) \in \mathbb{S}_{p,q}(G)} \text{cnt}_{p - |R_U|, q - |R_V|}(S_U, S_V)
\]

In agorithm \ref{alg:framework}, \shadow is initialized as $\{(\emptyset,\emptyset,U,V )\}$ and continously refined until stopping condition is met (Lines1--6). Which means that while $R_U$ and $R_V$ expand $S_U$ and $S_V$ shrink which result in increasing the bicilique denisity in the sampling space.

\red{give an example}

\begin{lemma}
	Lines 1--6 of Algorithm~\ref{alg:framework} correctly construct a valid BC-Shadow according to Definition~3.1.
\end{lemma}

\begin{proof}
	We prove by induction that after each iteration, the set $\mathbb{S}$ satisfies the conditions of Definition~3.1. Initially, $\mathbb{S} = \{ (\emptyset, \emptyset, U, V) \}$, which satisfies the conditions: since $R_U = \emptyset$ and $R_V = \emptyset$, $R_U \times R_V$ trivially satisfies the biclique condition (no edges exist to violate it); every vertex in $S_U = U$ is adjacent to all vertices in $R_V = \emptyset$ (vacuously true), and similarly for $S_V$ and $R_U$; and every $(p,q)$-biclique in $G$ is contained within $R_U \cup S_U = U$ and $R_V \cup S_V = V$. For the inductive step, assume $\mathbb{S}$ satisfies Definition~3.1 before refinement. During refinement, for each edge $(u, v) \in E$ with $u \in S_U$ and $v \in S_V$, we create a new subspace $(R_U', R_V', S_U', S_V')$ where $R_U' = R_U \cup \{ u \}$ and $R_V' = R_V \cup \{ v \}$; since $u$ is adjacent to $v$, and by the adjacency properties of $S_U$ and $S_V$, $R_U' \times R_V'$ forms a biclique. The updated shadow sets $S_U'$ and $S_V'$ consist of vertices adjacent to all of $R_V'$ and $R_U'$, satisfying the second condition. The uniqueness condition holds because any $(p,q)$-biclique extending from $(R_U, R_V)$ will be captured in exactly one of the new subspaces based on the ordering of vertices, ensuring no biclique is missed or duplicated. Therefore, after each iteration, $\mathbb{S}$ remains a valid BC-Shadow, and Lines 1--6 correctly construct it.
\end{proof}





\subsection{A Novel Sampling Structure}
After completing the BC-Shadow construction, the sampling space is significantly reduced. To count the bicliques, we can sample a subspace $\subspace$ and then sample an element $P$ from it to check $P$ forms a biclique. A simple approach might be to choose two vertex sets from $U$ and $V$ with sizes $p-|R_U|$ and $q-|R_V|$ respectively, but this naive method can still result in a large sampling space, \ie,$\binom{|S_U|}{p-|R_U|} \cdot \binom{|S_V|}{q-|R_V|}$. To further refine the process and make sampling more efficient, we need to impose additional constraints on $P$. In this section, we introduce a novel sampling structure, called zstar \cite{zigzag}, designed to efficiently sample elements from subspaces $(S_U,S_V,R_U,R_V) \in$ \shadow

For the convenience and improve the presentation of the general idea we introduce the concepts considering $G$ as the processing graph rather than the subspace $(S_U,S_V,R_U,R_V)$. Note that when applying for the proposed techniques to the subspace $(S_U,S_V,R_U,R_V)$ we need to consider the subgraph $G'(U',V',E')$ induced by $R_U,R_V$, $p$ as $p'$ where $p' = p-|R_U|$ and $q$ as $q'$ where $q' = q-|R_V|$.

For convenience, we assume without loss of generality that the vertices on each side of the bipartite graph $G'$ are arranged in some order.  Specifically, we have $u_1 \prec u_2 \prec \dots \prec u_{n_1}$ and $v_1 \prec v_2 \prec \dots \prec v_{n_2}$.We also assume that the $p \leq q$ because the same methods can be used when $p < q$. please note that for convenience, given examples in this paper does not consider the natural \red{is this correct?} ordering.

\begin{definition}
	\label{def:zstar}
	Given a bipartite graph $G(U, V, E)$ and integers $p,q,h$ , a $h$-zstar in $G$,is defined as an ordered vertex set
	\[
	Z = \{ u_{i_1}, v_{j_1}, u_{i_2}, v_{j_2}, \dots, u_{i_{h}}, v_{j_{h}}, \dots, v_{j_{h + (q - p)}} \}
	\]
	that satisfies the following conditions:
	\begin{enumerate}
		\item The number of vertices from the $U$ side is $h$, and the number of vertices from the $V$ side is $h + (q - p)$. \ie, $|Z|=2h+q-p$
		\item The vertices are ordered such that $i_1 < i_2 < \dots < i_{h}$ and $j_1 < j_2 < \dots < j_{h + (q - p)}$.
	\end{enumerate}
	Additionally, the length of a $(p, q)$-zstar is defined as $2h - 1$.
\end{definition}

 \begin{center}
	\begin{figure}
		
		\begin{tikzpicture}[scale=1, every node/.style={circle, draw, minimum size=0.5cm, inner sep=0pt, line width=0.8pt}]
			
			% Nodes
			\node[draw=blue] (u1) at (0,1.5) {$u_1$};
			\node[draw=blue]  (u2) at (1.3,1.5) {$u_2$};
			\node (u3) at (2.6,1.5) {$u_3$};
			\node (u4) at (3.9,1.5) {$u_4$};
			
			\node [draw=blue] (v1) at (0,0) {$v_1$};
			\node [draw=blue] (v2) at (1.3,0) {$v_2$};
			\node [draw=blue] (v3) at (2.6,0) {$v_3$};
			\node [draw=blue] (v4) at (3.9,0) {$v_4$};
			\node (v5) at (5.2,0) {$v_5$};
			% Edges
			\draw [draw=blue] (u1.south) -- (v1.north);
			\draw (u1.south) -- (v2.north);
			\draw (u1.south) -- (v3.north);
			\draw (u1.south) -- (v4.north);
			\draw [draw=blue] (u2.south) -- (v1.north);
			\draw [draw=blue] (u2.south) -- (v2.north);
			\draw [draw=blue] (u2.south) -- (v3.north);
			\draw [draw=blue] (u2.south) -- (v4.north);
			
			\draw (u3.south) -- (v2.north);
			\draw (u3.south) -- (v3.north);
			
			\draw (u3.south) -- (v4.north);
			\draw (u3.south) -- (v5.north);
			
			
			\draw (u4.south) -- (v4.north);
			\draw (u4.south) -- (v5.north);
			
		\end{tikzpicture}
		\caption{An example for $Z_{(2,4)}$}
		\label{fig:z1} 
	\end{figure}
\end{center}


\begin{center}
	\begin{figure}
		
		\begin{tikzpicture}[scale=1, 
			every node/.style={circle, draw, minimum size=0.5cm, inner sep=0pt, line width=0.8pt}
			]
			
			% Nodes
			\node (u1) at (0,1.5) {$u_1$};
			\node[draw=red] (u2) at (1.3,1.5) {$u_2$};
			\node [draw=red](u3) at (2.6,1.5) {$u_3$};
			\node (u4) at (3.9,1.5) {$u_4$};
			
			
			\node (v1) at (0,0) {$v_1$};
			\node[draw=red] (v2) at (1.3,0) {$v_2$};
			\node[draw=red] (v3) at (2.6,0) {$v_3$};
			\node [draw=red](v4) at (3.9,0) {$v_4$};
			\node [draw=red](v5) at (5.2,0) {$v_5$};
			
			% Edges
			\draw (u1.south) -- (v1.north);
			\draw (u1.south) -- (v2.north);
			\draw (u1.south) -- (v3.north);
			\draw (u1.south) -- (v4.north);
			\draw (u2.south) -- (v1.north);
			\draw[draw=red] (u2.south) -- (v2.north);
			\draw (u2.south) -- (v3.north);
			\draw (u2.south) -- (v4.north);
			
			\draw [draw=red](u3.south) -- (v2.north);
			\draw [draw=red](u3.south) -- (v3.north);
			
			\draw [draw=red](u3.south) -- (v4.north);
			\draw [draw=red](u3.south) -- (v5.north);
			
			
			\draw (u4.south) -- (v4.north);
			\draw (u4.south) -- (v5.north);
			
		\end{tikzpicture}
		\caption{Another for $Z_{(2,4)}$}
		\label{fig:z2} 
	\end{figure}
\end{center}

\red{edit examples}
Figures \ref{fig:z1} and \ref{fig:z2} shows two $2$-zstars in the example graph $G$.Note that the $2$-zstar in figure \ref{fig:z1} forms a (2,3)-bicique while figure \ref{fig:z2} does not form a (2,3)-biclique. In $2$-zstar shown in the figure \ref{fig:z1}, The primary purpose of designing zstar is to establish a correspondence between a biclique and a zstar such that every biclique has \textbf{exactly one} corresponding zstar. Note that every zstar may not have a corresponding biclique.

%\red{remove below}
%Let $\setbiclique(G) $ and $\Delta_{(p,q)}(G)$ denote  set of all \bicliques and  set of \zstars respectively in $G$ (we omit the subscript when the context is  clear), then the biclique density of $G$ is  calculated as
%
%\begin{equation}
%	\mathcal{\hat{B}}(G)  = \frac{cnt_{(p,q)}}{|\Delta|}
%\end{equation}

 let $\setzstar$ be the set of $h$-zstars in $G$.  We define a mapping $f: \setbiclique \to \Delta$ such that $f(b) = z$, where $B \in \setbiclique$ is a biclique and $Z \in \Delta$ is a \zstar. This mapping $f$ ensures that:

$$
\forall B \in \setbiclique, \exists Z \in \Delta \text{ such that } f(B) = Z,
$$

but

$$
\exists Z \in \Delta \text{ such that } \nexists B \in \setbiclique \text{ with } f(B) = Z.
$$

\begin{lemma}
	\label{lemma:zstar}
	given a $(p,q)$-biclique in $G$ with $p \leq q$, it contains exactly one \zstar if $p = h$
\end{lemma}

\begin{proof}
	The size of (p,q)-biclique is $p+q$ and by definition the size of \zstar is $2h+q-p$ .Hence a \biclique has exactly one \zstar.
\end{proof}

let $\mathcal{T}$ is the set of uniform \zstars sampled form $\Delta$ .


\begin{lemma}
	\label{lemma:zstarprob}
	let Z be a \zstar uniformly sampled from $\Delta$. The probability of $Z$ forms a  $(p,q)$-biclique in $G$ is $1/|\Delta|$
\end{lemma}

\begin{proof}
	As given in \ref{lemma:zstar} a \biclique contains exactly one \zstar,hence the lemma holds.
\end{proof}

Let $X(Z)$ be a binary random variable indicating whether a given \zstar Z forms a biclique, such that $X(Z) \in \{0,1\}$. For any Z, we define:

\[
X(Z) = \begin{cases}
	1 & \text{if } Z \text{ forms a biclique} \\
	0 & \text{otherwise}
\end{cases}
\]

Based on Lemma \ref{lemma:zstarprob}, we can construct an unbiased estimator for the count of bicliques by uniformly sampling Z from $\Delta$.


\begin{theorem}
 Consider $\mathcal{T}$, a set of $h$-zstars sampled uniformly from $\Delta$. The unbiased estimator for the count of $(p,q)$-bicliques in graph $G$, denoted as $\widehat{cnt}_{(p,q)}$, is given by:

	\[
	\estcnt = \frac{|\Delta| \sum_{Z \in \mathcal{T}} X(Z)}{|\mathcal{T}|}
	\]

	\begin{proof}[Proof]
		We proceed as follows:
		\begin{enumerate}
			\item By definition, we can establish that $\sum_{Z \in \mathcal{T}} X(Z) = \widehat{cnt}_{(p,q)}$
			\item This leads us to $\mathbb{E}[X(Z)] = \frac{\widehat{cnt}_{(p,q)}}{|\Delta|}$.
			\item Utilizing these results, we can demonstrate:
			\[
			\begin{aligned}
				\mathbb{E}\left[\estcnt\right] &= \mathbb{E}\left[\frac{|\Delta| \sum_{Z \in \mathcal{T}} X(Z)}{|\mathcal{T}|}\right] \\
				&= \frac{|\Delta|}{|\mathcal{T}|} \sum_{Z \in \mathcal{T}} \mathbb{E}[X(Z)] \\
				&= \cnt
			\end{aligned}
			\]
		\end{enumerate}
		Thus,  $\estcnt$ is an unbiased estimator for the count of $(p,q)$-bicliques.
	\end{proof}
\end{theorem}

\begin{algorithm}
	\label{algo:zstardp}
	\DontPrintSemicolon

	\caption{zstarDP$(G, p, q)$}
	\label{algo:zstardp}
	\KwIn{A bipartite graph $G(U, V, E)$ and integers $p$ and $q$}
	\KwOut{Dynamic programming table $dp$}
	\For{\textbf{each}  $e(v, u) \in E$}{
		Let $N_{>v}(u) \gets$ the set of neighbors of $u$ with vertex rank higher than the rank of $v$\;
		$dp[2][e(v, u)] \gets \binom{|N_{>v}(u)|}{q-p+1 }$\;
	}
	\For{$i = 3$ \textbf{to} $2h - 1$}{

		\For{\textbf{each}  $(v, u) \in E$}{
			\For{\textbf{each}  $(u', v)$ \textbf{where} $u' < u$}{
				$dp[i+1][(u',v)] \gets dp[i][(u',v)] + dp[i-1][e(v, u)]$\;
			}
		}

		\If{ $i \neq 2h-1$}{
			\For{\textbf{each}  $e(u', v) \in E$}{
				\For{\textbf{each}  $e(v', u')$ \textbf{where} $v' < v$}{
					$dp[i+1][e(v', u')] \gets dp[i+1][e(v', u')] + dp[i][e(u', v)]$\;
				}
			}

	}
		$i \gets i + 2$\;
	}

	\Return{$dp$};
\end{algorithm}


\textbf{Counting and Sampling zstars:} Reminding that we assume that $p \leq q$ and $h = p$, To maximize the utility of zstars, it is essential to efficiently count the number of \zstars in a graph $G$. We propose an efficient dynamic programming approach for counting \zstars, inspired by \cite{zigzag} . As defined in \ref{def:zstar}, If every zstar originates from a node $u$ , then the total count of zstars can be determined by summing the zstars that start from each $u$ in the set $U$.Let $dp[i][(u,v)]$ represent the number of zstars with length $i$ starting from $(u, v)$, where $i \in [2,2h-1] $. Then the number of zstars with length $2h-1$ starting from $u$ can be represented as $\sum_{v \in N(u)} dp[2h - 1][e(u, v)]$. Then the total zstars in $G$ is $\sum_{e \in E} dp[2h - 1][e(u, v)]$.

We noticed that $dp[i][e(u, v)]$ can also be computed using a dynamic programming algorithm. The core concept of our DP approach is that each zstar with the length of $i$ beginning from $u$ is built upon the zstar of length $i - 1$ starting from node $v$ when i > 2.Therefore, we have below recursion equation:

\begin{equation}
	\label{dpcount}
	\begin{cases}
		dp[i][e(u, v)] \gets \sum_{u' \in N_{> u}(v)} dp[i - 1][e(v, u')], \\
		dp[i - 1][e(v, u')] \gets \sum_{v' \in N_{> v}(u')} dp[i - 2][e(u', v')], \\
		dp[2][e(v, u)] \gets \binom{|N_{>u}(v)|}{q - p} \\
	\end{cases}
\end{equation}


Note that notations  $dp[i][e(u, v)] $ and $dp[i][e(v, u)] $ in equation (\ref{dpcount}) are different to each other. Algorithm \ref{algo:zstardp} outlines the steps for counting $h$-zstars. The algorithm takes as input the graph $G$ and two integers, $p$ and $q$, and produces a dynamic programming table that records the zstar counts for each edge in $G$, with zstar lengths ranging from 2 to $2h-1$.


The initial time complexity of Algorithm \ref{algo:zstardp} is $ O(h\cdot d_{max} \cdot |E|) $, where $ d_{max} $ is the maximum degree of the input graph $G$ and $ h = \min(p, q) $. This is inefficient for large graphs since $ d_{max} $ could be equal to number of vertices in either side, making the algorithm impractical for large graphs. To address this, we apply a differential-interval updating technique that reduces the time complexity to $ O(h \cdot |E|) $.

Consider the dynamic programming table at level $ i $, denoted as $ dp[i] $, represented as a sequence $ \{a_1, a_2, \dots, a_n\} $, where $ a_j = dp[i][e_j] $ corresponds to an edge $ e_j $. We transform this into a difference sequence $ \{b_1, b_2, \dots, b_n\} $, where $ b_1 = a_1 $ and $ b_j = a_j - a_{j-1} $ for $ j = 2 $ to $ n $. Each $ a_j $ is reconstructed by computing the prefix sum: $ a_j = \sum_{k=1}^{j} b_k $.

In the algorithm, updates occur when $ i $ is odd (for edges $ e(v, u') $ where $ u' > u $) and when $ i $ is even (for edges $ e(u, v') $ where $ v' > v $). These updates increment $ a_j $ over specific intervals. By using the difference sequence $ \{b_1, b_2, \dots, b_n\} $, we can efficiently apply these interval updates by modifying $ b_l \mathrel{+}= w $ and $ b_{r+1} \mathrel{-}= w $ for the range $[l, r]$, where $w$ is the value we need to add. After applying all updates, the original $ dp[i] $ is reconstructed by computing prefix sums.

This technique reduces the per-update time from $ O(k) $, where $ k $ is the number of edges in the interval, to $ O(1) $, leading to a total time complexity of $ O(h \cdot |E|) $.

\textbf{sampling a zstar}: Since we have efficient method to count the zstars in $G$ ,now we need to efficiently sample a zstar uniformly at random from $\setzstar$, ensuring that each zstar has an equal probability $\frac{1}{|\setzstar|}$ of being selected. To achieve this, we utilize the DP table $dp[i][e(u,v)]$, which records the number of partial zstars of length $i$ ending at edge $e(u,v)$.


We present an efficient method to uniformly sample a $h$-zstar from the bipartite graph $G$, leveraging the dynamic programming counts obtained earlier. The central challenge in uniformly generating a $h$-zstar is to establish the correct probability distribution over all possible zstars in $G$. By utilizing the counts computed in the dynamic programming table, we can construct this probability distribution effectively.

Recall that the total number of $h$-zstars in $G$ is given by:

\begin{equation} |\setzstar| = \sum_{e(u,v) \in E} dp[2h - 1][e(u,v)], \end{equation}

where $dp[2h - 1][e(u,v)]$ denotes the number of zstars of length $2h - 1$ starting from edge $e(u,v)$. This total count allows us to define the probability of an edge $e(u,v)$ being the starting edge of a zstar $Z$ as:

\begin{equation} P(e(u,v)) = \frac{dp[2h - 1][e(u,v)]}{\sum_{e(u,v) \in E} dp[2h - 1][e(u,v)]} \end{equation}

Once the starting edge is selected, the subsequent edges in the zstar can be determined recursively. Specifically, for the next edge $e(v,u')$ where $u' \in N_{>u}(v)$ (the neighbors of $v$ with a higher rank than $u$), the conditional probability of choosing $e(v,u')$ given that $e(u,v)$ is the current edge is:

\begin{equation} P(e(v,u') \mid e(u,v)) = \frac{dp[2h - 2][e(v,u')]}{dp[2h - 1][e(u,v)]}. \end{equation}

\noindent This recursive method continues for each subsequent edge in the zstar, decrementing $ i $ from $ 2h - 1 $ down to $ 2 $. By this stage, we have sampled $ p $ vertices from the $ U $ side and $ p - 1 $ vertices from the $ V $ side. To complete the h-zstar, we need to select the remaining $ q - p + 1 $ vertices from $ V $.

Given the last edge $ e(u, v) $ selected in the recursive procedure, we proceed by choosing $ q - p + 1 $ additional vertices from the set $N_{>v}(u)$. To maintain uniformity, each vertex $ v' $ in $ N_{>v}(u) $ is selected with equal probability:

\begin{equation}
	P(v') = \frac{1}{\left| N_{>v}(u) \right|}.
\end{equation}

By uniformly sampling these vertices, we ensure that all possible combinations are equally likely, thus preserving the uniformity of the overall sampling process. We summarise these steps in algorithm \ref{algo:samplingzstar}.




%
%\begin{algorithm}
%	\DontPrintSemicolon
%	\label{algo:samplingzstar}
%	\caption{Sampling a ZStar}
%	\KwIn{A bipartite graph $G=(U,V,E)$, integesr $h,q$}
%	\KwOut{}
%	Set the distribution $\mathcal{D}$ over edges where $p(e(u,v)) = \frac{dp[2h-1][e(u,v)]}{\sum e(u,v) dp[2h-1][e(u,v)]};$ \\
%	Sample an edge $e(u,v)$ according to $\mathcal{D}$;\\
%	$Z \gets \{e(u,v)\};$ \\
%	\For{$i = 2h-2$ \textbf{to} $2$}{
%		$E' \gets \{e(v,u') | u' \in N(v) \wedge u' > u\};$ \\
%		Set the distribution $\mathcal{D}$ over $E'$ where $p(e) = \frac{dp[i][e]}{\sum_{e \in E'} dp[i][e]};$ \\
%		Sample an edge $e(v,u')$ according to $D$; \\
%		$Z \gets Z \cup \{e(v,u')\};$ \\
%
%
%		$E' \gets \{e(u',v') | v' \in N(u') \wedge v' > v \};$ \\
%		Set the distribution $D$ over $E'$ where $p(e) = \frac{dp[i-1][e]}{\sum_{e \in E'} dp[i-1][e]};$ \\
%		Sample an edge $e(u',v')$ according to $\mathcal{D}$; \\
%		$Z \gets Z \cup \{e(u',v')\};$ \\
%		$u \gets u';$ $v \gets v';$\\
%		$i \gets i-2;$\\
%	}
%		$E' \gets \{e(u',v'') | v'' \in N(u') \wedge v'' > v' \};$ \\
%	\For{$j = q-h$ \textbf{to} $1$}{
%		Set the distribution $\mathcal{D}$ over $E'$ where $p(e) = \frac{1}{|E'|};$ \\
%		Sample an edge $e(u',v'')$ according to $\mathcal{D}$ \\
%		$E' \gets E'  \;  \backslash  \; e(u',v'')$
%	}
%	\Return{$Z;$}
%\end{algorithm}

\begin{algorithm}[]
	\DontPrintSemicolon
	\caption{Sampling a Z-Star}
	\label{algo:samplingzstar}
	\KwIn{A bipartite graph $G=(U,V,E)$, integers $p$, $q$}
	\KwOut{A uniformly sampled zstar $Z$}
	Set the distribution $\mathcal{D}$ over edges where $p(e(u,v)) = \frac{dp[2h-1][e(u,v)]}{\sum_{e(u,v) \in G} dp[2h-1][e(u,v)]};$ \\

	Sample an edge $e(u_0, v_0)$ according to $\mathcal{D}$\;
	Initialize $Z \gets \{ u_0, v_0 \}$, $k \gets 0$ \;

	\tcp{Recursive construction of zstar}
	\For{$i = 2h - 2$ \textbf{down to} $2$ \textbf{step} $-1$}{
		\eIf{$i$ is even}{
			\tcp{Sampling edge from $V$ to $U$}
			$E' \gets \{ e(v_k, u') \mid u' \in N_{>u_k}(v_k) \}$\;
			Set the distribution $\mathcal{D}$ over $E'$ where:
			\[
			P(e(v_k, u')) = \dfrac{dp[i][e(v_k, u')]}{dp[i+1][e(u_k, v_k)]}
			\]
			Sample an edge $e(v_k, u_{k+1})$ according to $\mathcal{D}$\;
			Append $u_{k+1}$ to $Z$\;

		}{
			\tcp{Sampling edge from $U$ to $V$}
			$E' \gets \{ e(u_{k+1}, v') \mid v' \in N_{>v_k}(u_{k+1}) \}$\;
			Set the distribution $\mathcal{D}$ over $E'$ where:
			\[
			P(e(u_{k+1}, v')) = \dfrac{dp[i][e(u_{k+1}, v')]}{dp[i+1][e(v_k, u_{k})]}
			\]
			Sample an edge $e(u_{k+1}, v_{k+1})$ according to $\mathcal{D}$\;
			Append $v_{k+1}$ to $Z$\;
			$k \gets k + 1$\;
		}

	}

	\tcp{Add remaining vertices from $V$}
	$V_{\text{rem}} \gets \{ v' \in N_{>v_k}(u_{k+1}) \}$\;
	Uniformly select $q - p + 1$ vertices from $V_{\text{rem}}$ without replacement; \\
	Append the selected vertices to $Z$\;
	\Return{$Z$}
\end{algorithm}





\subsection{Sampling Stopping Condition}

\red{writting as a generalised version, should it be written specifically for the sampling structure.}
\label{sec:sampling-stopping-condition}

In this subsection, we focus on the accuracy of our estimation, which is determined by \textbf{Stage II} of our algorithm.


The existing algorithm \cite{zigzag} requires the user to specify the number of samples $t$ to be drawn uniformly at random from $\sampelspace$. As this value is set manually,It cannot provide any formal guarantees on the accuracy of the estimates. To address this issue, we propose a different strategy for determining when to stop the sampling process. Specifically, we fix the required number$ s $ of successful samples $s$ (i.e., samples that correspond to actual$ (p,q) $-bicliques), and continue sampling from $ \sampelspace $ until $ s $ successful samples have been collected.We then estimate the biclique count $\cnt$ using $\estcnt = |\sampelspace| \cdot \frac{s}{t} $, where $t$ is the total number of samples drawn.

Based on the stopping rule theorem \cite{stoppingrule}, the estimation accuracy is guaranteed when the number of successful samples $s$ satisfies $ s \geq \gamma $, where
\[
\gamma = 1 + \frac{4(1+\epsilon)(e -2)\ln(2/\delta)}{\epsilon^2},
\]
where $ e $ is Euler's number. We will use the parameter $ \gamma $ for the quantity $1 + \frac{4(1+\epsilon)(e -2)\ln(2/\delta)}{\epsilon^2}$ throughout the remainder of the paper.

\begin{algorithm}[]
	\DontPrintSemicolon
	\caption{BicliqueEstimation($ \sampelspace, p, q, \epsilon, \delta $)}
	\label{alg:sr-biclique-estimator}
	\KwIn{$\sampelspace$, integers $ p, q $, error parameters $ \epsilon \in (0,1) $ and $ \delta \in (0,1) $}
	\KwOut{An estimate $ \estcnt $ of $ \cnt $}
	$ s \leftarrow 0; \quad t \leftarrow 0; $

	$ \gamma \leftarrow 1 + \dfrac{4(1+\epsilon)(e -2)\ln(2/\delta)}{\epsilon^2}; $

	\While{$ s < \gamma $}{
		Sample an element $A$ u.a.r. from $\sampelspace $;\;
		\If{ $A$ forms a $(p,q)$-biclique in $G$}{
			$ s \leftarrow s + 1; $
		}
		$ t \leftarrow t + 1; $
	}
	\Return$ \estcnt \leftarrow |\sampelspace| \cdot \dfrac{s}{t}; $
\end{algorithm}

\begin{theorem}
	Let $ \estcnt $ be the estimate returned by Algorithm~\ref{alg:sr-biclique-estimator}. Then, $P\left( |\estcnt - \cnt| > \epsilon \cdot \cnt \right) \leq \delta$.and the expected number of samples $ E[t] $ satisfies $\dfrac{\gamma \cdot |\sampelspace|}{\cnt} \leq E[t] < \dfrac{(\gamma + 1) \cdot |\sampelspace|}{\cnt}$.

\end{theorem}

\begin{proof}
Since each sample from $ \sampelspace $ is drawn uniformly at random, and $\dfrac{s}{t} $ is the proportion of successful samples, $\dfrac{s}{t}$ is an unbiased estimator of the biclique density $ \mu_{\sampelspace} = \dfrac{\cnt}{|\sampelspace|} $.

By applying the stopping rule theorem (e.g., see \cite{stoppingrule}), we have that the estimator $ \estcnt = |\sampelspace| \cdot \dfrac{s}{t} $ satisfies the desired accuracy guarantee when $ s \geq \gamma $.

The bounds on $ \mathbb{E}[t] $ follow from standard properties of negative binomial sampling, where the expected number of trials to achieve $ s $ successes in Bernoulli trials with success probability $ \mu_{\sampelspace} $ is $ E[t] = \dfrac{s}{\mu_{\sampelspace}} $. Substituting $ \mu_{\sampelspace} = \dfrac{\cnt}{|\sampelspace|} $ and $ s \geq \gamma $ yields the desired bounds.

\end{proof}

%
%\begin{proof}
%	Each sample from $ \sampelspace $ is drawn uniformly at random, and the proportion of successful samples $ \frac{s}{t} $ is an unbiased estimator of the biclique density $ \mu_{\sampelspace} = \frac{\cnt}{|\sampelspace|} $. Therefore, the estimator $ \estcnt = |\sampelspace| \cdot \frac{s}{t} $ is unbiased, i.e., $ \mathbb{E}[ \estcnt ] = \cnt $.
%
%	Applying the stopping rule theorem \cite{stoppingrule}, when we collect $ s \geq \gamma $ successful samples, the probability that $ \estcnt $ deviates from $ \cnt $ by more than $ \epsilon \cdot \cnt $ is at most $ \delta $:
%	\[
%	\Pr\left( | \estcnt - \cnt | > \epsilon \cdot \cnt \right) \leq \delta.
%	\]
%
%	For the expected number of samples $ E[t] $, since the number of trials needed to achieve $ s $ successes in Bernoulli trials with success probability $ \mu_{\sampelspace} $ follows a negative binomial distribution, we have:
%	\[
%	E[t] = \frac{s}{\mu_{\sampelspace}} = \frac{s \cdot |\sampelspace|}{\cnt}.
%	\]
%	Using $ s \geq \gamma $ and $ s < \gamma + 1 $, we obtain the bounds:
%	\[
%	\frac{\gamma \cdot |\sampelspace|}{\cnt} \leq E[t] < \frac{(\gamma + 1) \cdot |\sampelspace|}{\cnt}.
%	\]
%	This completes the proof.
%\end{proof}

The accuracy guarantee remains valid regardless of the specific sample space $ \sampelspace $, provided it includes all possible $ (p,q) $-bicliques in $ G $. However, different constructions of $ \sampelspace $ will influence the biclique density $ \mu_{\sampelspace} $, which in turn impacts the expected running time of the algorithm. In the following subsection, we will explore methods for constructing $ \sampelspace $ that aim to optimize the overall performance of the algorithm by balancing the computational workload between \textbf{Stage I} (sample space construction) and \textbf{Stage II} (sampling and estimation).




%	\section*{Balancing the Running Time of the Two Stages}
%
%
	\begin{algorithm}[]
		\DontPrintSemicolon

		\caption{BC-ShadowConstruction($G (U,V,E),p,q,\epsilon,\gamma$)}
		\label{algo::BC-ShadowConstruction}
		\KwIn{ Bipartite Graph $G (U,V,E)$ integers $p,q$ and error parameters $\epsilon,\delta$}
		\KwOut{Shadow $\mathbb{S}$}
		Arrange $U$ and $V$ in some order

	 $ \tcnt_{p,q}(G)\gets 1; $   $|\sampelspace| \gets |\mathcal{P}_{(p,q)}(G)|$ ;
		$\tilde{T}_{sample} \gets \infty$ \\
		$\mathbb{S} \gets  \{(\emptyset,\emptyset,U,V,\tcnt_{p,q}(G) / |\sampelspace|)\}$

		\While{ \text{ElapsedTime()}  $ < \gamma \cdot \frac{|\sampelspace| }{\tcnt_{(p,q)}(G)} \cdot \tilde{T}_{sample} $    }{

			($R_U,R_V,S_U,S_V,\ddot{\mu}) \gets \text{arg min}_{(R'_U,R'_V,S'_U,S'_V,\ddot{\mu}') \in \mathbb{S}}$ $ \ddot{\mu}';$ \\
			Remove  ($R_U,R_V,S_U,S_V,\ddot{\mu}) $ from $\mathbb{S};$
			$\tcnt_{(p,q)}(G) \gets \tcnt_{(p,q)}(G) -  |\mathcal{P}_{(p,q)}(S_U,S_V)| \cdot \ddot{\mu};$\\
			$|\sampelspace)| \gets |\sampelspace| - \mathcal{P}_{(p,q)}(S_U,S_V);$

			\If{$R_U = \emptyset$}{ $n_{sample} \gets 0; T_{total} \gets 0 $ }

			\For{\textbf{each} $ u \in S_U$}{
				\For{\textbf{each} $ v \in N_u(S_V)$}{
					($R'_U,R'_V,S'_U,S'_V,\ddot{\mu}$) $\gets$  ($R_U \cup u , R_V \cup v, N_v(S_U),N_u(S_V) $); \\
					$\ddot{\mu}' \gets \text{auxiliary $(p-|R'_U|,q-|R'_V|)$-biclique density in $(S'_U,S'_V)$ }$ from $\hat{n}$ samples; \\
					Add ($R'_U,R'_V,S'_U,S'_V,\ddot{\mu}'$) to $\mathbb{S};$ \\
					$\tcnt_{(p,q)}(G) \gets \tcnt_{(p,q)}(G) + |\mathcal{P}_{(p-|R'_U|,q-|R'_V|)}(S'_U,S'_V)| \cdot \ddot{\mu}'; $ \\
					$|\sampelspace| \gets  |\sampelspace| +  |\mathcal{P}_{(p-|R'_U|,q-|R'_V|)}(S'_U,S'_V)|;$\\
					$S_V \gets S_V \backslash v;$

					\If{ $S_U = \emptyset$}{
						$n_{sample} \gets n_{sample} + \hat{n}$ \\
						$T_{total} \gets T_{total}$ + running time of Line 13
					}
				}
				$S_U \gets S_U \backslash u;$
			}

		}
		\Return{$\mathbb{S}$}
	\end{algorithm}
%
%
%	In our biclique counting algorithm, achieving optimal efficiency hinges on balancing the computational efforts between Stage I (sample space construction) and Stage II (sampling to estimate the biclique count). An imbalance, where one stage consumes significantly more time than the other, can lead to unnecessary computational overhead and inefficient resource utilization.
%
%	If we spend too little time refining the sample space in Stage I, the sample space $\sampelspace$ remains large and contains many non-biclique elements, resulting in a low biclique density $ \mu_{S_{p,q}(G)} $. Consequently, Stage II will require a larger number of samples to achieve the desired estimation accuracy, increasing the total running time. Conversely, over-refinement in Stage I can lead to excessive computational time without proportionate gains in reducing the sample space or increasing the biclique density.
%
%	To strike a balance, we propose a strategy that dynamically decides when to stop refining the sample space based on the estimated running time of Stage II. The goal is to minimize the total running time by ensuring that the time spent in both stages is approximately equal. The expected running time of Stage II depends on the number of samples required, determined by the desired estimation accuracy, and the average time per sample, which includes the time taken to draw a sample and verify if it forms a biclique.
%
%
%	\[
%	\gamma = 1 + \frac{4(1+\epsilon)(e - 2)\ln(2/\delta)}{\epsilon^2},
%	\]
%
%	where $ \epsilon $ and $ \delta $ are the relative error and failure probability parameters, respectively, and $ e $ is Euler's number. The biclique density of the sample space is $ \mu_{S_{p,q}(G)} = \frac{\text{cnt}_{p,q}(G)}{|S_{p,q}(G)|} $, and $ T_{\text{sample}} $ represents the average time to sample an element from $ S_{p,q}(G) $ and check if it forms a biclique.
%
%	According to the stopping rule of our estimator (Algorithm 3), we require at least $ \gamma $ successful samples—that is, samples that are actual bicliques—to guarantee the estimation accuracy. The expected total number of samples $ t $ needed is $ t \approx \frac{\gamma}{\mu_{S_{p,q}(G)}} $. Therefore, the expected running time of Stage II is:
%
%	\[
%	\text{Expected Running Time of Stage II} = t \times T_{\text{sample}} = \frac{\gamma \times T_{\text{sample}}}{\mu_{S_{p,q}(G)}}.
%	\]
%
%	Our balancing strategy involves estimating $ \mu_{S_{p,q}(G)} $ and $ T_{\text{sample}} $ during the sample space refinement in Stage I. We decide to stop refining the sample space when the expected additional time spent in Stage I is no longer justified by the decrease in Stage II's running time. Formally, we stop refining when:
%
%	\[
%	\text{Elapsed Time in Stage I} \geq \frac{\gamma \times T_{\text{sample}}}{\tilde{\mu}},
%	\]
%	where $ \tilde{\mu} $ is the estimated biclique density of the current sample space.
%
%	To estimate $ \tilde{\mu} $, we use auxiliary sampling during Stage I. For each subspace $ (R_U, R_V, S_U, S_V) $ in the BC-Shadow $ S_{p,q}(G) $, we draw a small number of random samples from $ P_{p', q'}(S_U, S_V) $, where $ p' = p - |R_U| $ and $ q' = q - |R_V| $. We calculate the proportion of these samples that form valid bicliques to estimate the subspace density $ \mu' $. We then compute $ \tilde{\mu} $ as a weighted average of the subspace densities:
%
%	\[
%	\tilde{\mu} = \frac{\sum\limits_{(R_U, R_V, S_U, S_V)} |P_{p', q'}(S_U, S_V)| \times \mu'}{\sum\limits_{(R_U, R_V, S_U, S_V)} |P_{p', q'}(S_U, S_V)|}.
%	\]
%
%	To estimate $ T_{\text{sample}} $, we measure the total time taken during the auxiliary sampling to draw and verify samples, and divide it by the number of samples to get the average sampling time:
%
%	\[
%	T_{\text{sample}} = \frac{\text{Total Sampling Time}}{\text{Number of Samples}}.
%	\]
%
%	To effectively increase $ \mu_{S_{p,q}(G)} $ and reduce the expected number of samples in Stage II, we prioritize refining subspaces with the lowest estimated biclique densities $ \mu' $, as improving these will have the most significant impact on increasing the overall sample space density. After each refinement, we update $ \tilde{\mu} $ and reassess the stopping condition, ensuring that we capture the diminishing returns of further refinements.
%
%	The stopping condition for Stage I is thus:
%
%	\[
%	\text{Elapsed Time in Stage I} \geq \frac{\gamma \times T_{\text{sample}}}{\tilde{\mu}}.
%	\]
%
%	This condition ensures that we halt the refinement process when additional efforts in Stage I no longer result in proportionate decreases in Stage II's running time. By balancing the computational load between the two stages, we achieve an efficient overall algorithm.
%
%	By adopting this balancing strategy, we optimize resource utilization by preventing one stage from becoming a bottleneck, improve scalability by enabling the algorithm to handle larger graphs without excessive computations in either stage, and maintain estimation accuracy within a reasonable total running time. In summary, balancing the running time between Stage I and Stage II is crucial for the practical efficiency of our biclique counting algorithm. By estimating the expected running time of Stage II and dynamically adjusting the refinement process in Stage I, we achieve a harmonious balance that minimizes the total computational effort without compromising accuracy.


\subsection{Balancing the Running Time of the Two Stages}



If we spend minimal time refining the sample space in Stage I, the sample space $ S_{p,q}(G) $ remains large and contains many elements that do not form bicliques. This results in a low biclique density $ \mu_{S_{p,q}(G)} $, causing Stage II to require a large number of samples to achieve the desired estimation accuracy. Conversely, if we over-invest time in refining $ S_{p,q}(G) $, Stage I becomes time-consuming without proportionate benefits, as the gains in increasing $ \mu_{S_{p,q}(G)} $ diminish with each refinement.

To address this, we propose a strategy that dynamically balances the running time between the two stages by determining an optimal point at which to stop refining the sample space and begin sampling. This balance ensures that neither stage disproportionately dominates the total running time, optimizing the algorithm's overall efficiency.

\subsubsection{Estimating the Expected Running Time of Stage II}

The expected running time of Stage II depends on:
\begin{itemize}
	\item \textbf{The number of samples required}: Determined by the desired estimation accuracy and the biclique density of the sample space.
	\item \textbf{The average time per sample} $ T_{\text{sample}} $: Time taken to draw a sample from $ S_{p,q}(G) $ and verify whether it forms a biclique.
\end{itemize}

Following from the stopping rule theorem, the expected number of samples $ t $ taken when the sampling algorithm terminates is approximately:

\[
t \approx \frac{\gamma}{\mu_{S_{p,q}(G)}}
\]

where $ \gamma = 1 + \frac{4(1+\epsilon)(e - 2)\ln(2/\delta)}{\epsilon^2} $, $ \epsilon $ and $ \delta $ are the relative error and confidence parameters, and $ e $ is Euler's number.

Therefore, the expected running time of Stage II is:

\[
\text{Expected Running Time of Stage II} \approx \frac{\gamma}{\mu_{\mathcal{S}_{p,q}(G)}} \times T_{\text{sample}}
\tag{8}
\]

This equation indicates that as the biclique density $ \mu_{S_{p,q}(G)} $ increases, the expected running time of Stage II decreases, and vice versa.

\subsubsection{Effect of Refinement on Biclique Density}

We observe that each refinement of a sample subspace of the BC-Shadow $ S_{p,q}(G) $ reduces the size of the corresponding sample space and, consequently, increases the biclique density $ \mu_{S_{p,q}(G)} $. This is formalized in the following lemma.

\begin{lemma}
	Assuming that the same ordering of vertices is used in defining $ P_{p-|R_U|, q-|R_V|}(S_U, S_V) $ for different subspaces of $ S_{p,q}(G) $, and that vertices are processed in the same order during refinement, each refinement of a sample subspace of $ S_{p,q}(G) $ results in a smaller sample space and an increased biclique density.
\end{lemma}

\begin{proof}
	Suppose a subspace $ (R_U, R_V, S_U, S_V) $ is refined into \\ $\{(R_{U_i}, R_{V_i}, S_{U_i}, S_{V_i})\}_{i=1}^l $ by the refinement process. We need to show that:

	\[
	\bigcup_{i=1}^l S_{p,q}(R_{U_i}, R_{V_i}, S_{U_i}, S_{V_i}) \subseteq S_{p,q}(R_U, R_V, S_U, S_V)
	\]

	This means that the refined sample spaces are subsets of the original sample space. Additionally, since the sample spaces do not overlap due to the ordering, we have:

	\[
	S_{p,q}(R_{U_i}, R_{V_i}, S_{U_i}, S_{V_i}) \cap S_{p,q}(R_{U_j}, R_{V_j}, S_{U_j}, S_{V_j}) = \emptyset \quad \text{for } i \neq j
	\]

	Thus, each refinement reduces the size of the sample space, and since the number of bicliques remains the same or decreases less rapidly, the biclique density $ \mu_{S_{p,q}(G)} $ increases.
\end{proof}

\subsubsection{Balancing Strategy}

If we stop Stage I immediately after initializing $ S_{p,q}(G) $ as $ \{(\emptyset, \emptyset, U, V)\} $, Stage I will be very efficient, but Stage II will require a long time due to the low biclique density. Conversely, extensive refinement in Stage I will increase $ \mu_{S_{p,q}(G)} $ but also significantly increase the time spent in Stage I.

To determine when to stop refining and proceed to Stage II, we need to estimate the expected running time of Stage II without actually executing it. We can compute $ \gamma $ since $ \epsilon $ and $ \delta $ are input parameters, and $ T_{\text{sample}} $ can be estimated by sampling a few elements from $ S_{p,q}(G) $ and measuring the average time. However, estimating $ \mu_{S_{p,q}(G)} $ is challenging since it is the very quantity we aim to estimate.To address this challenge, we propose computing an auxiliary estimation $ \tilde{\mu} $ for $ \mu_{S_{p,q}(G)} $. Unlike the final estimate, $ \tilde{\mu} $ does not require strict theoretical accuracy guarantees, as its sole purpose is to influence the running time, rather than the accuracy of the final biclique count $\cnt$. However, in order to compute $ \tilde{\mu} $ effectively, certain conditions must be met. Specifically, we require a sampling structure where each biclique in the graph corresponds uniquely to a sampling structure, ensuring that $ \tilde{\mu} $ remains within the range $ [0, 1] $.  Our novel sampling structure facilitates this, enabling us to maintain this condition and thereby compute $\tilde{\mu}$ accurately and efficiently.

\subsubsection{Construction Stopping Condition}

Our construction stopping condition is:

\[
\text{Elapsed Time in Stage I} \geq \gamma \times \frac{ \tilde{T}_{\text{sample}}}{\tilde{\mu}}
\]

where $ \tilde{T}_{\text{sample}} $ is the estimated average time per sample, and $ \tilde{\mu} $ is the auxiliary estimation of the biclique density.

\subsubsection{Computing the Auxiliary Estimation $ \tilde{\mu} $}

As the sample space $ S_{p,q}(G) $ changes dynamically during refinement, we need to continuously estimate $ \tilde{\mu} $. For each subspace $ (R_U, R_V, S_U, S_V) \in S $, we estimate the biclique density $ \mu' $ within that subspace by:

\begin{itemize}
	\item Sampling a small number of elements uniformly at random from $ P_{p', q'}(S_U, S_V) $, where $ p' = p - |R_U| $ and $ q' = q - |R_V| $.
	\item Calculating $ \mu' $ as the proportion of these samples that form bicliques in $ G $.
	\item Estimating the biclique count in the subspace as $ \text{cnt}_{p',q'}(S_U, S_V) \approx |P_{p', q'}(S_U, S_V)| \times \mu' $.
\end{itemize}

We store $ \mu' $ along with the subspace and update $\tcnt_{(p,q)}(G)$ and sample space size accordingly:

\[
\tcnt_{(p,q)}(G) = \sum_{(R_U, R_V, S_U, S_V, \mu') \in S} |P_{p', q'}(S_U, S_V)| \times \mu'
\]

\[
|S_{p,q}(G)| = \sum_{(R_U, R_V, S_U, S_V) \in S} |P_{p', q'}(S_U, S_V)|
\]

The auxiliary estimation of the biclique density is then:

\[
\tilde{\mu} = \frac{\text{cnt}_{p,q}(G)}{|\sampelspace|}
\]

\subsubsection{The Pseudocode of Shadow Construction}

Based on the above discussions, the pseudocode of our shadow construction algorithm for biclique counting is presented in Algorithm~\ref{algo::BC-ShadowConstruction}. The algorithm takes a bipartite graph $G(U, V, E)$, integers $p$, $q$, and error parameters $\epsilon$, $\delta$ as input, and outputs a shadow $\mathbb{S}$.

The initialization phase begins by arranging the vertices in $U$ and $V$ in some order to facilitate efficient sampling and refinement (Line 1). The initial count $\tcnt_{p,q}(G)$ is set to 1, and the size of the sample space $|S_{p,q}(G)|$ is initialized to $|\mathcal{P}_{p,q}(G)|$ (Lines 2–3). The value of $\tilde{T}_{\text{sample}}$ is initially set to infinity, and the shadow $\mathbb{S}$ is initialized with the full vertex sets $U$ and $V$, along with the initial estimation of $\mu$ (Line 4).

In the refinement loop (Lines 5–24), the algorithm continues refining the shadow as long as the elapsed time is less than the estimated running time of Stage II (Line 5). At each iteration, the subspace $(R_U, R_V, S_U, S_V, \mu)$ with the smallest $\mu$ is selected for refinement (Line 6). The algorithm updates $\tcnt_{p,q}(G)$ and the size of the sample space $|S_{p,q}(G)|$ by removing the contribution of the selected subspace (Lines 7–8). For each $u \in S_U$ and $v \in N_u(S_V)$, new subspaces are created by extending $R_U$ and $R_V$, and auxiliary information is computed to estimate the biclique density $\mu'$ in the new subspaces using a small sample size (Lines 11–13). These new subspaces are added back to the shadow $\mathbb{S}$, and the values of $\tcnt_{p,q}(G)$ and $|S_{p,q}(G)|$ are updated accordingly (Lines 14–15). The processed vertices are removed from $S_U$ and $S_V$ (Lines 16–17), and after processing the initial subspaces, the algorithm estimates $\tilde{T}_{\text{sample}}$ based on the total sampling time and number of samples (Lines 21–22). This approach ensures efficient refinement of the subspaces while maintaining computational feasibility during the sampling process.



%
%\begin{algorithm}[]
%	\DontPrintSemicolon
%
%	\caption{BicliqueEstimation2($\mathbb{S},p,q,\epsilon,\delta$)}
%	\label{BicliqueEstimation}
%	\KwIn{ Shadow $\mathbb{S} $ integers $p,q $, error parameters$\epsilon,\delta$ and a rough esitmation for the $(p,q)$-biclique density $  \tilde{\mu}$ }
%	\KwOut{$\tcnt_{(p,q)}(G)$ }
%
%	$s \gets 0; s_{est} \gets 0; t \gets 0$ \\
%	$t_b \gets \frac{\gamma}{\tilde{\mu}}$ \\
%	constuct alias structure for sampling sub spaces \\
%	\While{ $s < \gamma $   }{
%
%%		\For{\textbf{each} subsapce ($R_U,R_V,S_U,S_V) \in \mathbb{S}$} {
%%			initialize $c_{(R_U,R_V,S_U,S_V)} \gets 0$
%%		}
%	sample a subspace (($R_U,R_V,S_U,S_V,\ddot{\mu}))  \in \mathbb{S}$ with probability $\frac{\mathcal{P}_{(p-|R_U|,q-|R_V|)}(S_U,S_V)}
%	{|\samplespace|}$
%%		\For{$i \gets 1$ to $t_b$}{
%%
%%
%%			$c_{(R_U,R_V,S_U,S_V)} \gets   c_{(R_U,R_V,S_U,S_V)} + 1 $
%%		}
%
%		\For{\textbf{each} subsapce ($R_U,R_V,S_U,S_V) \in \mathbb{S}$ with  $c_{(R_U,R_V,S_U,S_V)} > 0$} {
%			\red{how to construct the alias structure here}
%
%			\For{$i \gets 1$ to $c_{(R_U,R_V,S_U,S_V)} > 0$}{
%				sample an element P u.a.r. from  $|\mathcal{P}_{(p-|R_U|,q-|R_V|)}(S_U,S_V)$\\
%
%				\If{ P formas a biclique in G} {
%					$s \gets s + C_{(p-|R_U|,q-|R_V|)} /$ \red{todo here}
%					$s_{est} \gets s_{est} + C_{(p-|R_U|,q-|R_V|)} $
%				}
%				$t \gets t + 1$
%			}
%		}
%
%	}
%	\red{correct this}\\
%	$\tcnt_{(p,q)}(G) \gets  |\samplespace|  \cdot \frac{s}{t} $\\
%	\Return{$\tcnt_{(p,q)}(G) $}
%\end{algorithm}


\subsection{Estimating Bicliques}



\begin{algorithm}[]
	\DontPrintSemicolon

	\caption{BicliqueEstimation2($\mathbb{S},p,q,\epsilon,\delta$)}
	\label{bcefinal}
	\KwIn{ Shadow $\mathbb{S} $ integers $p,q $, error parameters $\epsilon,\delta$ }
	\KwOut{$\tcnt_{(p,q)}(G)$ }

$s \gets 0;\ t \gets 0$\;

$\gamma = 1 + \frac{4(1+\epsilon)(e -2)\ln(2/\delta)}{\epsilon^2}$\;

\While{$s < \gamma$}{
	Sample a subspace $(R_U, R_V, S_U, S_V, \ddot{\mu}) \in \mathbb{S}$ with probability $\dfrac{|\mathcal{P}_{(p - |R_U|, q - |R_V|)}(S_U, S_V)|}{|\sampelspace|}$\;

	Sample an element $P$ uniformly at random from $\mathcal{P}_{(p - |R_U|, q - |R_V|)}(S_U, S_V)$\;

	\If{$P$ forms a biclique in $G$}{
		$s \gets s + 1$\;
	}
	$t \gets t + 1$\;
}
$\estcnt \gets |\sampelspace| \cdot \dfrac{s}{t}$\;
\Return{$\estcnt$}

\end{algorithm}



The final algorithm for estimating the number of $(p,q)$-bicliques is shown in Algorithm~\ref{bcefinal}.The algorithm proceeds by repeatedly sampling elements from the overall sample space until the number of successful samples $s$ reaches the threshold $\gamma$ (Lines 3 8). In each iteration, a subspace $(R_U, R_V, S_U, S_V, \ddot{\mu})$ is selected from the shadow $\mathbb{S}$ with probability proportional to the size of its corresponding sampling space $\mathcal{P}_{(p - |R_U|, q - |R_V|)}(S_U, S_V)$ relative to the total sample space size $|\sampelspace|$ (Line 4). This weighted sampling ensures that every potential $(p,q)$-biclique in the total sample space has an equal probability of being selected, maintaining uniformity across the entire sample space.

Once a subspace is selected, an element $P$ is sampled uniformly at random from $\mathcal{P}_{(p - |R_U|, q - |R_V|)}(S_U, S_V)$ (Line 5). This element represents a $p-|R_U|$-zstar.We then check whether the sampled element $P$ forms a biclique in $G$ (Line 6). If it does, we increment the successful sample counter $s$ (Line 7). Regardless of whether $P$ forms a biclique, we increment the total sample counter $t$ (Line 8). This process continues until the number of successful samples $s$ reaches the threshold $\gamma$ (Line 3), ensuring that sufficient evidence has been gathered to achieve the desired estimation accuracy.After obtaining the required number of successful samples, we compute the estimate $\estcnt$ of the total number of $(p,q)$-bicliques (Line 9). This estimate is calculated by scaling the ratio of successful samples to total samples by the size of the overall sample space $|\sampelspace|$.Finally, the algorithm returns this estimate $\estcnt$ (Line 10), which, due to the properties of the sampling method and the stopping condition, satisfies the accuracy and confidence guarantees specified by the error parameters $\epsilon$ and $\delta$.




\begin{acks}
 This work was supported by the [...] Research Fund of [...] (Number [...]). Additional funding was provided by [...] and [...]. We also thank [...] for contributing [...].
\end{acks}

%\clearpage

\bibliographystyle{ACM-Reference-Format}
\bibliography{references}

\end{document}
\endinput
